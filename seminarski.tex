% !TEX encoding = UTF-8 Unicode

\documentclass[a4paper]{article}

\usepackage{color}
\usepackage{url}
\usepackage[T2A]{fontenc}
\usepackage[utf8]{inputenc}
\usepackage{graphicx}

\usepackage[english,serbian]{babel}

\usepackage[unicode]{hyperref}
\hypersetup{colorlinks,citecolor=green,filecolor=green,linkcolor=blue,urlcolor=blue}

\begin{document}

\title{Mešanje poverljivih transakcija: Sveobuhvatna privatnost Bitcoin transakcija\\ \small{Seminarski rad u okviru kursa\\Kriptografija\\ Matematički fakultet}}

\author{Miloš Samardžija\\mi13304@alas.matf.bg.ac.rs}

\maketitle

\abstract {Ovaj tekst predstavlja sažetak rada \textit{Mixing Confidential Transactions: Comprehensive Transaction Privacy for Bitcoin} \cite{bitcoinprivacy}. Ukratko su prikazani problemi sa privatnošću Bitcoin transakcija, njihova delimična rešenja, poteškoće sa kompatibilnošću koje se javljaju prilikom spajanja ovih rešenja, i kako ih protokol VallueShuffle prevazilazi. Opisani su gradivni blokovi ovog protokola, kao i neke od njegovih karakteristika.}
\tableofcontents

\newpage

\section{Uvod}
\label{sec:uvod}

\addcontentsline{toc}{section}{Literatura}
\appendix
\bibliography{seminarski} 
\bibliographystyle{plain}
\end{document}
